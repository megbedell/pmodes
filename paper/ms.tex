\documentclass[modern]{aastex62}

\usepackage{graphicx}
\usepackage{xcolor}
\usepackage{xspace}
\usepackage[sort&compress]{natbib}
\usepackage[hang,flushmargin]{footmisc}


% style tweaks
\newcommand{\acronym}[1]{{\small{#1}}}
\newcommand{\project}[1]{\textsl{#1}}
\newcommand{\code}[1]{{\texttt{#1}}}
\newcommand{\todo}[1]{\textcolor{red}{#1}}

% the following is stolen from Adrian Price-Whelan (github.com/adrn/latex-init):
\usepackage{hyperref}
\definecolor{niceblue}{rgb}{0.0, 0.4, 0.65}
\definecolor{linkcolor}{rgb}{0.02,0.35,0.55}
\definecolor{citecolor}{rgb}{0.4,0.4,0.4}
\hypersetup{colorlinks=true,linkcolor=linkcolor,citecolor=citecolor,
            filecolor=linkcolor,urlcolor=linkcolor}
\hypersetup{pageanchor=false}

% astronomy
\newcommand{\teff}{\ensuremath{T_{\rm eff}}}
\newcommand{\logg}{\ensuremath{\log g}}
\newcommand{\feh}{\ensuremath{\mathrm{[Fe/H]}}}
\newcommand{\vt}{\ensuremath{v_t}}
\newcommand{\mh}{\ensuremath{\mathrm{[M/H]}}}
\newcommand{\xh}{\ensuremath{\mathrm{[X/H]}}}
\newcommand{\I}{\textsc{I}}
\newcommand{\II}{\textsc{II}}
\newcommand{\vsini}{\ensuremath{v \sin{i}}}
\newcommand{\gcm}{\ensuremath{\mathrm{g}~\mathrm{cm}^{-3}}}
\newcommand{\kms}{\ensuremath{\mathrm{km}~\mathrm{s}^{-1}}}
\newcommand{\masyr}{\ensuremath{\mathrm{mas}~\mathrm{yr}^{-1}}}
\newcommand{\msun}{\ensuremath{\mathrm{M}_\odot}}
\newcommand{\ang}{\text{\normalfont\AA}}


\newcommand{\TF}{\code{TensorFlow}\xspace}
\newcommand{\python}{\code{python}\xspace}
\newcommand{\HARPS}{\project{\acronym{HARPS}}\xspace}
\newcommand{\HIRES}{\project{\acronym{HIRES}}\xspace}
\newcommand{\RV}{\acronym{RV}\xspace}
\newcommand{\EPRV}{\acronym{EPRV}\xspace}


% stolen from Ben Pope:
\newcommand{\kepler}{\emph{Kepler}\xspace}
\newcommand{\hipparcos}{\emph{Hipparcos}\xspace}
\newcommand{\gaia}{\emph{Gaia}\xspace}
\newcommand{\ktwo}{\emph{K2}\xspace}

% misc shortcuts
\newcommand{\flatiron}{Flatiron Institute, Simons Foundation, 162 Fifth Ave, New York, NY 10010, USA}
\newcommand{\chicago}{Department of Astronomy and Astrophysics, University of
Chicago, 5640 S. Ellis Ave, Chicago, IL 60637, USA}
\newcommand{\nyuccpp}{Center for Cosmology and Particle Physics, Department of Physics, New York University, 726 Broadway, New York, NY 10012, USA}
\newcommand{\nyucds}{Center for Data Science, New York University, 60 Fifth Ave, New York, NY 10011, USA}
\newcommand{\mpia}{Max-Planck-Institut f\"ur Astronomie, K\"onigstuhl 17, 69117 Heidelberg, Germany}


\shorttitle{Working Title}
\shortauthors{Bedell et al.}

\setlength{\parindent}{1.4em} % trust in Hogg
\begin{document}\sloppy\sloppypar\raggedbottom\frenchspacing % trust in Hogg

\graphicspath{ {figures/} }
\DeclareGraphicsExtensions{.pdf,.eps,.png}

\title{Optimal Radial Velocity Observing Strategies for Solar-Like Oscillators}

\author[0000-0001-9907-7742]{Megan Bedell}
\affiliation{\flatiron}

\author{David W. Hogg}

\author{Daniel Foreman-Mackey}

\author{Sam Grunblatt}


\correspondingauthor{Megan Bedell}
\email{mbedell@flatironinstitute.org}


\begin{abstract}\noindent
\todo{to be written}
% context
% aims
% methods
% results
\end{abstract}

\section{Introduction}
\label{s:intro}

There are many sources of noise in \EPRV arising from stellar variability on different timescales. 
Modern observing programs are engineered to minimize the contributions of such signals to the final \RV time series.

One tall pole in the \EPRV error budget is asteroseismic oscillations. 
Previous work has been done on mitigating these via optimal integration timescales.

There are, however, downsides to integrating over these oscillations. 
Instead we might want to \textit{resolve} and model the asteroseismic signals.

In this work, we use a combination of realistic simulations and actual \HARPS data to explore the potential benefits of observing strategies for \EPRV which resolve stellar p-mode oscillations. 


\section{Simulated Data}

\begin{itemize}
\item Power spectra used: simple solar-like oscillator with p-modes only and with granulation too
\item (two-panel plot of power spectra)
\item How WF's GP model works
\item Timestamps of observations
\item (two-panel plot of full, noise-free observations)
\item Making these into realistic observations: how we handle binning + noise injections (both photon + read noise)
\end{itemize}

\section{Observing Strategies}

\subsection{Long Exposures}

Test performance of integrating over p-modes. Show that due to stochasticity, an ``optimal'' exposure time isn't always optimal.

\subsection{Short Exposures}

Test performance of resolving p-modes. Say some things about Nyquist limits to motivate choice of exposure times.

\begin{itemize}
\item Functional form of model used
\item Performance in several regimes of read noise/read-out time
\item Does this still work with granulation included?
\end{itemize}

\subsection{Hybrid Strategies}

???

\section{Validation on Real Data}

\HARPS observations of asteroseismic star -- probably 18 Sco \citep{Bazot2012}.

\section{Conclusions}

Recommendations for observers.

Potential limitations -- will other noise sources leak in? Is there any chance that this will bias recovery of planet signals?


\acknowledgements
The authors thank Will Farr, Rodrigo Luger, Simon J. Murphy, Benjamin Pope, Didier Queloz, and Lily Zhao for useful conversations.

\software{
    \code{Astropy} \citep{astropy},
    \code{IPython} \citep{ipython},
    \code{matplotlib} \citep{matplotlib},
    \code{numpy} \citep{numpy},
    \code{scipy} (\url{https://www.scipy.org/}),
}

% \facility{}

\clearpage
\bibliographystyle{aasjournal}
\bibliography{ms}

\end{document}