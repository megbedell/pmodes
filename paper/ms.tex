\documentclass[modern]{aastex62}
\graphicspath{ {figures/} }
\DeclareGraphicsExtensions{.pdf,.eps,.png}

\usepackage{graphicx}
\usepackage{xcolor}
\usepackage{xspace}
\usepackage[sort&compress]{natbib}
\usepackage[hang,flushmargin]{footmisc}


% style tweaks
\newcommand{\acronym}[1]{{\small{#1}}}
\newcommand{\project}[1]{\textsl{#1}}
\newcommand{\code}[1]{{\texttt{#1}}}
\newcommand{\todo}[1]{\textcolor{red}{#1}}

% the following is stolen from Adrian Price-Whelan (github.com/adrn/latex-init):
\usepackage{hyperref}
\definecolor{niceblue}{rgb}{0.0, 0.4, 0.65}
\definecolor{linkcolor}{rgb}{0.02,0.35,0.55}
\definecolor{citecolor}{rgb}{0.4,0.4,0.4}
\hypersetup{colorlinks=true,linkcolor=linkcolor,citecolor=citecolor,
            filecolor=linkcolor,urlcolor=linkcolor}
\hypersetup{pageanchor=false}

% astronomy
\newcommand{\teff}{\ensuremath{T_{\rm eff}}}
\newcommand{\logg}{\ensuremath{\log g}}
\newcommand{\feh}{\ensuremath{\mathrm{[Fe/H]}}}
\newcommand{\vt}{\ensuremath{v_t}}
\newcommand{\mh}{\ensuremath{\mathrm{[M/H]}}}
\newcommand{\xh}{\ensuremath{\mathrm{[X/H]}}}
\newcommand{\I}{\textsc{I}}
\newcommand{\II}{\textsc{II}}
\newcommand{\vsini}{\ensuremath{v \sin{i}}}
\newcommand{\gcm}{\ensuremath{\mathrm{g}~\mathrm{cm}^{-3}}}
\newcommand{\kms}{\ensuremath{\mathrm{km}~\mathrm{s}^{-1}}}
\newcommand{\masyr}{\ensuremath{\mathrm{mas}~\mathrm{yr}^{-1}}}
\newcommand{\msun}{\ensuremath{\mathrm{M}_\odot}}
\newcommand{\ang}{\text{\normalfont\AA}}


\newcommand{\TF}{\code{TensorFlow}\xspace}
\newcommand{\python}{\code{python}\xspace}
\newcommand{\HARPS}{\project{\acronym{HARPS}}\xspace}
\newcommand{\HIRES}{\project{\acronym{HIRES}}\xspace}
\newcommand{\RV}{\acronym{RV}\xspace}
\newcommand{\EPRV}{\acronym{EPRV}\xspace}


% stolen from Ben Pope:
\newcommand{\kepler}{\emph{Kepler}\xspace}
\newcommand{\hipparcos}{\emph{Hipparcos}\xspace}
\newcommand{\gaia}{\emph{Gaia}\xspace}
\newcommand{\ktwo}{\emph{K2}\xspace}

% misc shortcuts
\newcommand{\flatiron}{Flatiron Institute, Simons Foundation, 162 Fifth Ave, New York, NY 10010, USA}
\newcommand{\chicago}{Department of Astronomy and Astrophysics, University of
Chicago, 5640 S. Ellis Ave, Chicago, IL 60637, USA}
\newcommand{\nyuccpp}{Center for Cosmology and Particle Physics, Department of Physics, New York University, 726 Broadway, New York, NY 10012, USA}
\newcommand{\nyucds}{Center for Data Science, New York University, 60 Fifth Ave, New York, NY 10011, USA}
\newcommand{\mpia}{Max-Planck-Institut f\"ur Astronomie, K\"onigstuhl 17, 69117 Heidelberg, Germany}


\shorttitle{mitigation of p-mode noise}
\shortauthors{bedell \textit{et al.}}

\setlength{\parindent}{1.4em} % trust in Hogg
\begin{document}\sloppy\sloppypar\raggedbottom\frenchspacing % trust in Hogg

\graphicspath{ {figures/} }
\DeclareGraphicsExtensions{.pdf,.eps,.png}

\title{Optimal Radial Velocity Observing Strategies for Solar-Like Oscillators}


\author[0000-0001-9907-7742]{Megan Bedell}
\affiliation{\flatiron}


\author[0000-0003-2866-9403]{David W. Hogg}
\affiliation{\flatiron}
\affiliation{\nyuccpp}
\affiliation{\nyucds}
\affiliation{\mpia}

\author[0000-0002-9328-5652]{Dan Foreman-Mackey}
\affiliation{\flatiron}

\author{others}

\correspondingauthor{Megan Bedell}
\email{mbedell@flatironinstitute.org}

\begin{abstract}\noindent
% context
Decade-scale surveys taking extreme precision radial-velocity (\EPRV)
measurements provide our best hope for finding and characterizing
truly Earth-like planets around nearby Sun-like stars.
The best \EPRV measurements are not currently limited by photon noise;
they are limited by the intrinsic variability of the target stars.
The simplest kind of variability, in some sense, is the variability
created by the fact that the target stars are, in general, pulsating
in asteroseismic pressure modes (p-modes).
% aims
Here we look at some of the practical and information-theoretic
considerations around different strategies for mitigating p-mode
noise.
% methods
We simulate the p-mode-induced intrinsic stellar surface variability
with a dynamical system that is a forest of modes, each of which is a
damped, driven harmonic oscillator, independently driven by white noise.
We consider p-mode mitigation strategies that involve choosing
sensible exposure times, including fine-tuned exposure times that
(nearly) null the dominant modes, short exposure times that resolve
the modes, and long exposure times that integrate over many mode
periods.
We consider data-analysis methods that treat the measurements as
independent measurements of the stellar velocity (simple chi-squared
fitting) and methods that model the noise as correlated in the way
that the mixture of p-mode contributions should be (kernel or
Gaussian-process methods).
% results
We find that, independent of the choice of exposure time, the
sensitivity of an \EPRV search or characterization project to
small-amplitude planet signals is improved if the data are analyzed
with an accurate p-mode noise model.
Furthermore, we find no advantage, in this data-analysis framework, to
choosing exposure times that are optimal for nulling the p-mode noise.
One limitation of these conclusions is that a good p-mode noise model
requires knowledge of the p-mode spectrum (mode frequencies and
amplitudes) this advocates taking some \EPRV or photometric data on
each target star at short enough cadences to resolve the modes; it's
a small price to pay for \emph{Earth 2.0}.
\end{abstract}

\section{Introduction}
\label{s:intro}

There are many sources of noise in \EPRV arising from stellar variability on different timescales. 
Modern observing programs are engineered to minimize the contributions of such signals to the final \RV time series.

One tall pole in the \EPRV error budget is asteroseismic oscillations. 
Previous work has been done on mitigating these via optimal integration timescales.

There are, however, downsides to integrating over these oscillations. 
Instead we might want to \textit{resolve} and model the asteroseismic signals.

In this work, we use a combination of realistic simulations and actual \HARPS data to explore the potential benefits of observing strategies for \EPRV which resolve stellar p-mode oscillations. 


\section{Simulated Data}

\begin{itemize}
\item Power spectra used: simple solar-like oscillator with p-modes only and with granulation too
\item (two-panel plot of power spectra)
\item How WF's GP model works
\item Timestamps of observations
\item (two-panel plot of full, noise-free observations)
\item Making these into realistic observations: how we handle binning + noise injections (both photon + read noise)
\end{itemize}

\section{Observing Strategies}

\subsection{Long Exposures}

Test performance of integrating over p-modes. Show that due to stochasticity, an ``optimal'' exposure time isn't always optimal.

\subsection{Short Exposures}

Test performance of resolving p-modes. Say some things about Nyquist limits to motivate choice of exposure times.

\begin{itemize}
\item Functional form of model used
\item Performance in several regimes of read noise/read-out time
\item Does this still work with granulation included?
\end{itemize}

\subsection{Hybrid Strategies}

???

\section{Validation on Real Data}

\HARPS observations of asteroseismic star -- probably 18 Sco \citep{Bazot2012}.

\section{Discussion}

Recommendations for observers.

Potential limitations -- will other noise sources leak in? Is there any chance that this will bias recovery of planet signals?

Need to know mode spectrum

The possibility that there are Teff variations that are observable.

\acknowledgements

The authors thank
  Ben Pope (NYU),
  Will Farr (Flatiron),
  Rodrigo Luger (Flatiron),
  Simon J. Murphy (Sydney),
  Didier Queloz (Cambridge), and
  Lily Zhao (Yale)
for useful conversations.

\software{
    \code{Astropy} \citep{astropy},
    \code{IPython} \citep{ipython},
    \code{matplotlib} \citep{matplotlib},
    \code{numpy} \citep{numpy},
    \code{scipy} (\url{https://www.scipy.org/}),
}

% \facility{}

\clearpage
\bibliographystyle{aasjournal}
\bibliography{ms}

\end{document}
